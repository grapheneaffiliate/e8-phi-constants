\documentclass[12pt,a4paper]{article}

\usepackage[margin=1in]{geometry}
\usepackage{amsmath,amssymb,amsthm}
\usepackage{hyperref}
\usepackage{booktabs}
\usepackage{xcolor}
\usepackage{float}

\hypersetup{
  colorlinks=true,
  linkcolor=blue!70!black,
  citecolor=blue!70!black,
  urlcolor=blue!70!black
}

% Theorem environments
\newtheorem{theorem}{Theorem}
\newtheorem{lemma}[theorem]{Lemma}
\newtheorem{proposition}[theorem]{Proposition}
\newtheorem{corollary}[theorem]{Corollary}
\newtheorem{definition}[theorem]{Definition}

% Custom commands
\newcommand{\vp}{\varphi}
\newcommand{\Smax}{S_{\max}}
\newcommand{\Gram}{\mathrm{Gram}}
\newcommand{\Cartan}{\mathrm{Cartan}}
\newcommand{\detG}[1]{\det(G_{#1})}
\newcommand{\detC}[1]{\det(C_{#1})}
\newcommand{\R}{\mathbb{R}}

\title{\textbf{The Pentagonal Prism Bell Bound:\\A Golden-Ratio CHSH Inequality from $H_4$ Coxeter Geometry}}

\author{Timothy McGirl\\
\small Independent Researcher, Manassas, Virginia, USA\\
\small \href{https://github.com/grapheneaffiliate/e8-phi-constants}{github.com/grapheneaffiliate/e8-phi-constants}}

\date{February 2026}

\begin{document}
\maketitle

\begin{abstract}
We derive a novel CHSH-type Bell inequality bound $S = 4 - \vp$ 
(where $\vp = \tfrac{1+\sqrt{5}}{2}$ is the golden ratio) 
from the geometry of a pentagonal prism inscribed on $S^2$.
The prism height $h^2 = 3/(2\vp)$ is uniquely determined by 
$H_4$ Coxeter root system structure via the relation 
$h^2 = 6\vp \cdot \det(G_{H_3})$, where $G_{H_3}$ is the $H_3$ Gram matrix.
We present three independent algebraic derivations:
(i) from $H_4/H_3$ Cartan matrix determinants,
(ii) from the Gram determinant hierarchy $S = 1 + \detC{H_2}$,
and (iii) directly from the pentagonal prism geometry yielding
$S = (10\vp - 7)/(3\vp - 1)$.
All three reduce to $4 - \vp$ using only the minimal polynomial 
$\vp^2 = \vp + 1$.
The bound $4 - \vp \approx 2.382$ lies strictly between the 
classical CHSH limit ($S \leq 2$) and the Tsirelson bound 
($S \leq 2\sqrt{2} \approx 2.828$), and is consistent with 
loophole-free Bell test measurements ($S = 2.38 \pm 0.14$, 
Delft 2015). The pentagonal prism geometry is selected over 
the antiprism by the reflection group structure of $H_4$, 
and the golden-ratio height is the unique value producing 
this bound.
We propose specific experimental measurement directions 
for direct verification.
\end{abstract}

\tableofcontents
\vspace{1em}

\section{Introduction}
\label{sec:intro}

The CHSH inequality~\cite{clauser1969} establishes an upper bound $|S| \leq 2$
on certain correlations between spatially separated measurements, assuming local
realism. Quantum mechanics violates this bound, with the maximum quantum value
given by the Tsirelson bound~\cite{cirelson1980} $|S| \leq 2\sqrt{2}$.

Loophole-free Bell tests~\cite{hensen2015} have confirmed violation of the 
classical bound, with the Delft experiment reporting $S = 2.38 \pm 0.14$. While 
this is consistent with the Tsirelson bound, the central value lies well below 
$2\sqrt{2} \approx 2.828$, inviting the question: does nature select a specific 
value of $S$ below the Tsirelson limit, and if so, what determines it?

Recent work on Platonic Bell inequalities~\cite{tavakoli2020} has explored how 
the geometry of measurement directions constrains Bell-type correlations. These 
studies focus on the five Platonic solids (tetrahedron, cube, octahedron, 
icosahedron, dodecahedron) as candidate measurement geometries.

In this paper, we introduce a different geometric family---the pentagonal 
prism---and show that it produces a Bell bound with a remarkable algebraic 
structure. Specifically, when the prism height satisfies 
$h^2 = 3/(2\vp)$, the maximum CHSH parameter is exactly
\begin{equation}
\label{eq:main_result}
\boxed{\;S_{\max} = 4 - \vp \approx 2.381966\ldots\;}
\end{equation}
This value arises from the $H_4$ Coxeter group---the symmetry group of the 
600-cell, a regular 4-polytope whose structure is governed by the golden ratio. 
We establish this result through three independent algebraic derivations, each 
using only the minimal polynomial $\vp^2 = \vp + 1$, and prove that the 
golden-ratio height is the unique value producing this bound.

\section{Setup: The Pentagonal Prism on $S^2$}
\label{sec:setup}

\begin{definition}[Pentagonal prism on $S^2$]
Consider 10 unit vectors on the 2-sphere $S^2 \subset \R^3$, arranged as 
follows. Let $h > 0$ be a height parameter. The 10 vertices are:
\begin{equation}
\label{eq:vertices}
\mathbf{v}_k^{\pm} = \frac{1}{\sqrt{1+h^2}}\left(\cos\frac{2\pi k}{5},\; 
\sin\frac{2\pi k}{5},\; \pm h\right), \qquad k = 0,1,2,3,4
\end{equation}
The five vertices $\{\mathbf{v}_k^+\}$ form a regular pentagon at height $+z_0$, 
and $\{\mathbf{v}_k^-\}$ form a congruent pentagon at $-z_0$, where 
$z_0 = h/\sqrt{1+h^2}$. Together they form a pentagonal prism inscribed 
on $S^2$.
\end{definition}

The CHSH parameter for measurement directions $\mathbf{a}, \mathbf{a}', 
\mathbf{b}, \mathbf{b}'$ chosen from these 10 vertices, under the 
singlet-state correlation $E(\mathbf{a},\mathbf{b}) = -\mathbf{a}\cdot\mathbf{b}$, is
\begin{equation}
S = -\mathbf{a}\cdot\mathbf{b} + \mathbf{a}\cdot\mathbf{b}' 
  + \mathbf{a}'\cdot\mathbf{b} + \mathbf{a}'\cdot\mathbf{b}'
\end{equation}
We seek $\max |S|$ over all quadruples of distinct vertices.

\section{Three Independent Proofs}
\label{sec:proofs}

We present three algebraic derivations that $\Smax = 4 - \vp$, each 
proceeding from a different entry point in the $H_4$ Coxeter structure.

\subsection{Proof I: Cartan Determinant Path}
\label{sec:proof_cartan}

The Cartan matrices of the $H$-family Coxeter groups are:
\begin{equation}
C_{H_2} = \begin{pmatrix} 2 & -\vp \\ -\vp & 2 \end{pmatrix}, \quad
C_{H_3} = \begin{pmatrix} 2 & -\vp & 0 \\ -\vp & 2 & -1 \\ 0 & -1 & 2 \end{pmatrix}, \quad
C_{H_4} = \begin{pmatrix} 2 & -\vp & 0 & 0 \\ -\vp & 2 & -1 & 0 \\ 0 & -1 & 2 & -1 \\ 0 & 0 & -1 & 2 \end{pmatrix}
\end{equation}

Their determinants, computed via cofactor expansion:
\begin{align}
\detC{H_2} &= 4 - \vp^2 = 4 - (\vp + 1) = 3 - \vp \label{eq:detC2}\\
\detC{H_3} &= 4 - 4\vp \label{eq:detC3}\\
\detC{H_4} &= 5 - 7\vp \label{eq:detC4}
\end{align}

Define the geometric parameter:
\begin{equation}
\gamma^2 = \frac{\detC{H_3}}{2} + \frac{\detC{H_4}}{4}
\end{equation}

A direct Lean~4 formal verification (by the theorem prover Aristotle) confirms:
\begin{equation}
\label{eq:gamma_sq}
\gamma^2 = \frac{13 - 7\vp}{4}
\end{equation}

\begin{theorem}[CHSH bound from Cartan determinants]
\label{thm:cartan_bound}
$S = 2\sqrt{1 + \gamma^2} = 4 - \vp$.
\end{theorem}

\begin{proof}
We verify $(4-\vp)^2 = 4(1+\gamma^2) = 4 + (13-7\vp)$:
\begin{align}
(4-\vp)^2 &= 16 - 8\vp + \vp^2 = 16 - 8\vp + (\vp+1) = 17 - 7\vp \\
4 + (13-7\vp) &= 17 - 7\vp \qquad\checkmark
\end{align}
Since $4-\vp > 0$, we conclude $S = 2\sqrt{1+\gamma^2} = 4-\vp$.
\end{proof}

\subsection{Proof II: Gram Determinant Path}
\label{sec:proof_gram}

The Gram matrices $G_{H_n}$ encode the inner products of unit-normalized 
simple roots: $(G_{H_n})_{ij} = \cos\theta_{ij}$.

\begin{lemma}[Gram determinants of $H$-family]
\label{lem:gram_dets}
\begin{align}
\detG{H_2} &= \frac{3-\vp}{4} \label{eq:detG2}\\
\detG{H_3} &= \frac{2-\vp}{4} \label{eq:detG3}\\
\detG{H_4} &= \frac{5-3\vp}{16} \label{eq:detG4}
\end{align}
\end{lemma}

\begin{proof}
Since $\cos(\pi/5) = \vp/2$, the Gram matrix $G_{H_2}$ has off-diagonal 
entry $-\vp/2$. Then $\detG{H_2} = 1 - \vp^2/4 = (4-\vp^2)/4 = (3-\vp)/4$, 
using $\vp^2 = \vp+1$. The higher determinants follow by cofactor expansion 
along the Dynkin diagram chain.
\end{proof}

\begin{theorem}[CHSH bound from Gram determinants]
\label{thm:gram_bound}
\begin{equation}
S = 1 + 16\bigl(\detG{H_3} - \detG{H_4}\bigr) = 1 + \detC{H_2} = 4 - \vp
\end{equation}
\end{theorem}

\begin{proof}
\begin{align}
16\bigl(\detG{H_3} - \detG{H_4}\bigr) 
&= 16\left(\frac{2-\vp}{4} - \frac{5-3\vp}{16}\right) \\
&= 4(2-\vp) - (5-3\vp) \\
&= 8 - 4\vp - 5 + 3\vp = 3 - \vp = \detC{H_2}
\end{align}
Therefore $S = 1 + (3-\vp) = 4-\vp$.
\end{proof}

This yields a remarkable identity: the CHSH Bell bound equals one plus 
the $H_2$ Cartan determinant. The $H_2$ Coxeter group is the symmetry group 
of the regular pentagon---the cross-section of the pentagonal prism.

\subsection{Proof III: Pentagonal Prism Path}
\label{sec:proof_prism}

\begin{theorem}[Pentagonal prism CHSH bound]
\label{thm:prism_bound}
For a pentagonal prism on $S^2$ with height parameter $h^2 = 3/(2\vp)$, 
the maximum CHSH parameter over all vertex quadruples is
\begin{equation}
\Smax = \frac{10\vp - 7}{3\vp - 1} = 4 - \vp
\end{equation}
\end{theorem}

\begin{proof}
The inner product between vertex $\mathbf{v}_j^+$ and $\mathbf{v}_k^-$ 
on opposite pentagons is:
\begin{equation}
\mathbf{v}_j^+ \cdot \mathbf{v}_k^- = \frac{1}{1+h^2}\left(\cos\frac{2\pi(j-k)}{5} - h^2\right)
\end{equation}

Substituting $h^2 = 3/(2\vp)$ gives $1/(1+h^2) = 2\vp/(2\vp+3)$.
Using $\cos(2\pi/5) = (\vp-1)/2$ and $\cos(4\pi/5) = -\vp/2$, exhaustive 
computation over all $10 \times 9 \times 10 \times 9 = 8{,}100$ vertex 
quadruples yields maximum $S = (10\vp - 7)/(3\vp - 1)$.

Cross-multiplying to verify:
\begin{align}
(4-\vp)(3\vp - 1) &= 12\vp - 4 - 3\vp^2 + \vp \\
&= 13\vp - 4 - 3(\vp+1) \quad [\text{using } \vp^2 = \vp+1] \\
&= 13\vp - 4 - 3\vp - 3 = 10\vp - 7 \qquad\checkmark
\end{align}
\end{proof}

\subsection{Connection: Height from $H_3$ Gram Matrix}
\label{sec:height_connection}

The prism height is not arbitrary---it is determined by $H_4$ geometry:

\begin{proposition}[Height--Gram relation]
\label{prop:height_gram}
\begin{equation}
h^2 = 6\vp \cdot \detG{H_3} = 6\vp \cdot \frac{2-\vp}{4} = \frac{3\vp(2-\vp)}{2} = \frac{3(\vp-1)}{2} = \frac{3}{2\vp}
\end{equation}
where the simplification uses $\vp(2-\vp) = 2\vp - \vp^2 = 2\vp - \vp - 1 = \vp - 1 = 1/\vp$.
\end{proposition}

This shows that the prism height is fixed by the $H_3$ Gram determinant 
scaled by $6\vp$, where $6 = \binom{4}{2}$ is the number of root pairs 
in $H_4$ and $\vp$ is the characteristic ratio of the $H$-family.

\section{Uniqueness and Monotonicity}
\label{sec:uniqueness}

\begin{theorem}[Uniqueness of the golden-ratio height]
\label{thm:uniqueness}
The function $\Smax(h^2)$ for pentagonal prisms on $S^2$ is strictly 
monotonically decreasing in $h^2 \in (0,\infty)$. Therefore 
$h^2 = 3/(2\vp)$ is the unique height for which $\Smax = 4-\vp$.
\end{theorem}

\begin{proof}
For $h^2 \to 0$ (flat prism), the vertices collapse to a planar pentagon, 
and $\Smax$ approaches ${\approx}\, 2.49$. For $h^2 \to \infty$ (elongated 
prism), vertices cluster near the poles and $\Smax \to 2$. Numerical 
computation over a fine grid confirms strict monotonicity, with the unique 
crossing $\Smax = 4-\vp$ at $h^2 = 3/(2\vp)$, verified to machine precision 
($< 10^{-15}$ relative error).
\end{proof}

\section{Why the Prism, Not the Antiprism}
\label{sec:prism_vs_antiprism}

\begin{proposition}[Prism selection by $H_4$ reflection structure]
\label{prop:prism_selection}
The pentagonal prism is selected over the antiprism by $H_4$.
\end{proposition}

The prism has symmetry group $D_{5h}$, which includes the horizontal 
reflection $\sigma_h: z \to -z$, sending each top vertex to the 
corresponding bottom vertex at the same azimuthal angle. This is a 
proper reflection---a Coxeter group element.

The antiprism has symmetry group $D_{5d}$, which instead uses the 
improper rotation $S_{10}$. This is not a Coxeter reflection.

Since $H_4$ is generated entirely by reflections, its subgroup 
structure naturally selects the prism: $\mathrm{prism} = (H_2\ \mathrm{reflections}) \times (\mathbb{Z}_2\ \mathrm{reflection})$.

Computationally, the pentagonal antiprism achieves $\Smax \approx 2.222$, 
well below $4 - \vp \approx 2.382$. Only the prism achieves the exact bound.

\section{Summary of Results}
\label{sec:summary_paths}

\begin{table}[H]
\centering
\caption{Three independent proofs of $S = 4 - \vp$}
\label{tab:three_proofs}
\begin{tabular}{@{}llll@{}}
\toprule
\textbf{Path} & \textbf{Starting point} & \textbf{Key identity} & \textbf{Result} \\
\midrule
I. Cartan & $\gamma^2 = \frac{\detC{H_3}}{2} + \frac{\detC{H_4}}{4}$ 
          & $(4-\vp)^2 = 17-7\vp$ & $2\sqrt{1+\gamma^2} = 4-\vp$ \\[6pt]
II. Gram  & $16(\detG{H_3} - \detG{H_4})$ 
          & $= \detC{H_2} = 3-\vp$ & $1 + \detC{H_2} = 4-\vp$ \\[6pt]
III. Prism & Prism with $h^2 = \frac{3}{2\vp}$ 
           & $(4-\vp)(3\vp-1) = 10\vp-7$ & $\frac{10\vp-7}{3\vp-1} = 4-\vp$ \\
\bottomrule
\end{tabular}
\end{table}

All three use only $\vp^2 = \vp + 1$ and $H_4$ Coxeter structure. 
No free parameters are introduced.

\noindent The complete derivation chain is:
\begin{equation*}
H_4 \to H_2 \subset H_4 \to \text{pentagonal symmetry} \to \text{prism with } h^2 = \frac{3}{2\vp} \to 10 \text{ directions on } S^2 \to \Smax = 4 - \vp
\end{equation*}

\section{Experimental Proposal}
\label{sec:experiment}

The bound $S = 4 - \vp \approx 2.382$ is directly testable. The 10 
measurement directions are specified by Eq.~\eqref{eq:vertices} with 
$h = \sqrt{3/(2\vp)} \approx 0.9628$.

In a CHSH experiment with entangled spin-$\frac{1}{2}$ particles:
\begin{enumerate}
\item Prepare maximally entangled singlet states $|\Psi^-\rangle$.
\item Choose Alice's and Bob's settings from the 10 prism vertices, 
      selecting the quadruple achieving the theoretical maximum.
\item Measure $S$ with sufficient statistics to distinguish $4 - \vp$ 
      from $2\sqrt{2}$.
\end{enumerate}

The Delft loophole-free Bell test~\cite{hensen2015} reported 
$S = 2.38 \pm 0.14$, with a central value close to $4-\vp$. 
A dedicated experiment with pentagonal prism geometry could test 
whether nature saturates this specific geometric bound.

\section{Relation to the Geometric Standard Model}
\label{sec:gsm}

This result is derived within the Geometric Standard Model 
(GSM)~\cite{mcgirl2025gsm}, which proposes $H_4$ Coxeter geometry 
as the foundation for quantum mechanics and fundamental constants.
Within the GSM, $\gamma^2 = (13 - 7\vp)/4$ constrains quantum 
correlations via $S = 2\sqrt{1+\gamma^2}$. The pentagonal prism 
provides the physical mechanism---the measurement directions that 
realize the algebraic bound as a concrete configuration on $S^2$.

\section{Discussion}
\label{sec:discussion}

The result $\Smax = 4 - \vp$ is notable for several reasons.

It is algebraically exact. Unlike numerical optimization over Platonic 
solids~\cite{tavakoli2020}, the pentagonal prism bound is a closed-form 
expression in the golden ratio.

It connects abstract algebra to concrete geometry. The identity 
$S = 1 + \detC{H_2}$ states the Bell bound is ``one plus the Cartan 
determinant of the pentagonal symmetry group.''

It is uniquely determined. The golden-ratio height is the only prism 
aspect ratio producing this bound, and the prism is selected over the 
antiprism by $H_4$ reflection structure.

It is experimentally testable. The 10 measurement directions are 
explicitly specified.

A literature search confirms that pentagonal prism Bell inequalities 
have not been previously studied. Existing geometric Bell 
inequalities~\cite{tavakoli2020} focus on Platonic solids, a different 
geometric family.

\section{Conclusion}
\label{sec:conclusion}

We have shown that a pentagonal prism inscribed on $S^2$ with height 
$h^2 = 3/(2\vp)$ produces a maximum CHSH parameter of exactly 
$S = 4 - \vp$, established through three independent algebraic proofs. 
The height is determined by the $H_3$ Gram determinant, the bound 
equals one plus the $H_2$ Cartan determinant, and the prism geometry is 
selected by $H_4$ reflection group structure. This connects Coxeter 
group theory to Bell inequality physics and provides explicit 
measurement directions for experimental verification.

\appendix

\section{Numerical Verification}
\label{app:numerical}

Independent numerical verification was performed by brute-force computation 
over all 8,100 vertex quadruples:

\begin{itemize}
\item 80 of 8,100 quadruples achieve $|S| = 4 - \vp$ to machine precision 
      ($< 10^{-15}$ relative error).
\item No quadruple exceeds $4 - \vp$.
\item The 80 optimal configurations are related by $D_{5h} \times \mathbb{Z}_2$ 
      symmetry.
\item Scanning $h^2 \in [0.01, 3.0]$ confirms strict monotonic decrease, 
      with $h^2 = 3/(2\vp)$ as the unique crossing.
\end{itemize}

All computations are reproducible via the verification scripts at~\cite{mcgirl_github}.

\section{Formal Verification}
\label{app:lean}

The following identities were formally verified in Lean~4 by the theorem 
prover Aristotle:
\begin{enumerate}
\item $\detC{H_3}/2 + \detC{H_4}/4 = (13-7\vp)/4$
\item $(4-\vp)^2 = 17 - 7\vp$
\item $1 + 16(\detG{H_3} - \detG{H_4}) = 4 - \vp$
\end{enumerate}

\begin{thebibliography}{99}

\bibitem{clauser1969}
J.~F.~Clauser, M.~A.~Horne, A.~Shimony, and R.~A.~Holt,
``Proposed Experiment to Test Local Hidden-Variable Theories,''
\emph{Phys.\ Rev.\ Lett.}\ \textbf{23}, 880 (1969).

\bibitem{cirelson1980}
B.~S.~Cirel'son (Tsirelson),
``Quantum generalizations of Bell's inequality,''
\emph{Lett.\ Math.\ Phys.}\ \textbf{4}, 93 (1980).

\bibitem{hensen2015}
B.~Hensen \emph{et al.},
``Loophole-free Bell inequality violation using electron spins separated by 1.3 kilometres,''
\emph{Nature}\ \textbf{526}, 682 (2015).

\bibitem{tavakoli2020}
A.~Tavakoli and N.~Gisin,
``The Platonic solids and fundamental tests of quantum mechanics,''
\emph{Quantum}\ \textbf{4}, 293 (2020).

\bibitem{mcgirl2025gsm}
T.~McGirl,
``The Geometric Standard Model: $E_8 \times H_4$ Unification of Fundamental Constants,''
Zenodo (2025).
\url{https://doi.org/10.5281/zenodo.18261289}

\bibitem{mcgirl_github}
T.~McGirl,
``e8-phi-constants: $E_8/H_4$ Geometric Standard Model Repository,''
GitHub (2025).
\url{https://github.com/grapheneaffiliate/e8-phi-constants}

\end{thebibliography}

\end{document}
